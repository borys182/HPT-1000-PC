%\usepackage{polski}
\usepackage{fontspec}	%pakiet umozliwiajacy uzycie czcionki Cambria
\setmainfont{Cambria}	%ustawienie głownej czcionki
\usepackage{unicode-math}
\usepackage{mempatch}
\usepackage{memhfixc}
\usepackage{color,graphicx}
\usepackage{ifthen}
\usepackage{imakeidx}
\makeindex
\setmathfont{Cambria Math}	%ustawienie czcionki do wyrazen matematycznych
\usepackage{color}	%pakiet umozliwiajacy uzywanie kolorow
\usepackage{colortbl}
\definecolor{gray}{rgb}{0.52,0.52,0.52}	%zdefiniowanie koloru szarego
\usepackage[left=2.5cm,right=2.5cm,top=2.5cm,footskip=2cm,bottom=2.5cm]{geometry}	%okreselenie rozmiarow marginesow
\usepackage{overpic}	%pakiet umozliwiajacy nakladanie na siebie obrazkow, tekstu itp. 	
\usepackage[pagestyles]{titlesec}	%pakiet do formatowania naglowkow, stopek, tytulow sekcji, rozdzialow - rozmiar, wyglad itp.
\titleformat{\chapter}{\fontsize{24}{28.8}\selectfont\bfseries}{\thechapter}{10pt}{\fontsize{24}{28.8}\selectfont\bfseries}	%formatowanie wygladu nazwy rozdzialu
\titlespacing{\chapter}{0pt}{0pt}{11pt}	%formatowanie odleglosci miedzy tytulem rozdzialu a tekstem z lewej, gory i dolu
\titleformat{\section}{\fontsize{16}{19.2}\selectfont\bfseries}{\thesection}{10pt}{\fontsize{16}{19.2}\selectfont\bfseries}	%formatowanie wygladu nazwy sekcji


\usepackage{booktabs}
\usepackage{array}
\renewcommand{\arraystretch}{1.5}
\setlength{\tabcolsep}{10pt}
\newcommand{\centercell}[1]{\multicolumn{1}{|c|}{#1}}
\usepackage{supertabular}  
\usepackage[xetex]{hyperref}
\usepackage{anyfontsize}
\setcounter{secnumdepth}{3}                                                                             
\usepackage{makeidx}
\makeindex
\usepackage[usenames,dvipsnames]{xcolor}
\usepackage{tcolorbox}
\usepackage{ltablex}
\usepackage{tabularx}
\usepackage{array}
\usepackage{colortbl}
\tcbuselibrary{skins}
\usepackage{graphicx}
\usepackage{colortbl,array}
\newcolumntype{Y}{>{\raggedleft\arraybackslash}X}
\newcolumntype{b}{>{\footnotesize}X}
\newcolumntype{s}{>{\footnotesize\hsize=.3\hsize}X}
\usepackage[yyyymmdd,hhmmss]{datetime}
\usepackage{stackengine}
\usepackage{scalerel}
\usepackage{xcolor}
\usepackage{multirow}
\usepackage{fancyhdr}
\usepackage{placeins}
\usepackage{hyperref}
