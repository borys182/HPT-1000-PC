\chapter{INSTALLATION}
\thispagestyle{fancy}

\section {SOFTWARE}

In order to install software run the \textit{ \SetupFileName file}. Installation
wizard will open. \\

\textbf{Notice:} Installation should be run with administrator privileges.

	\begin{figure}[!h] 
	\centering \includegraphics[width=0.5\textwidth]{Graphic/Installation/install_run_as_administrator.png}	
	\caption{Run install with administrator privileges}
	\label{run_install_with_administrator_privileges}
	\end{figure}
	\FloatBarrier

Follow the direction in the installation wizard.  

\begin{enumerate}
	\item Select destination location where should \textit{Quartz Monitor} be installed. \\
	\begin{figure}[!h] 
	\centering \includegraphics[width=0.7\textwidth]{Graphic/Installation/choose_location.png}	
	\caption{Select location of software}
	\label{select_location_of_software}
	\end{figure}
	\clearpage

	\item Select additional shortcuts for application on desktop. \\
	\begin{figure}[!h] 
	\centering \includegraphics[width=0.7\textwidth]{Graphic/Installation/desktop_shortcut.png}	
	\caption{Select additional shortcuts}
	\label{select_additional_shortcuts}
	\end{figure}
	\FloatBarrier

	\item Install application. \\
	\begin{figure}[!h] 
	\centering \includegraphics[width=0.7\textwidth]{Graphic/Installation/install.png}	
	\caption{Ready to install}
	\label{redy_to_install}
	\end{figure}
   \clearpage

	\item Application install in progress. \\
	\begin{figure}[!h] 
	\centering \includegraphics[width=0.7\textwidth]{Graphic/Installation/install_in_progress.png}	
	\caption{Install in progress}
	\label{install_in_progress}
	\end{figure}
	\FloatBarrier

	\item Install driver for USB. \\
	\begin{figure}[!h] 
	\centering \includegraphics[width=0.7\textwidth]{Graphic/Installation/Instal_driver_usb_exe.png}	
	\caption{Install driver USB}
	\label{install_in_progress}
	\end{figure}
	\FloatBarrier

	\item Extract required files \\
	\begin{figure}[!h] 
	\centering \includegraphics[width=0.7\textwidth]{Graphic/Installation/extract_required_files_driver_usb.png}	
	\caption{Extract required files for USB driver}
	\label{install_in_progress}
	\end{figure}
	\FloatBarrier

	\begin{figure}[!h] 
	\centering \includegraphics[width=0.7\textwidth]{Graphic/Installation/finish_extract_driver_usb.png}	
	\caption{Finished extract files driver USB}
	\label{install_in_progress}
	\end{figure}
	\FloatBarrier

	\newpage
	\item License agreement \\
	\begin{figure}[!h] 
	\centering \includegraphics[width=0.7\textwidth]{Graphic/Installation/license_agreement.png}	
	\caption{License agreement for Install  driver USB}
	\label{install_in_progress}
	\end{figure}
	\FloatBarrier

\item Install USB driver finished \\
	\begin{figure}[!h] 
	\centering \includegraphics[width=0.7\textwidth]{Graphic/Installation/finished_install_driver_usb.png}	
	\caption{Install  driver USB finished}
	\label{install_in_progress}
	\end{figure}
	\FloatBarrier

\newpage
	\item Install program finished. \\
	\begin{figure}[!h] 
	\centering \includegraphics[width=0.7\textwidth]{Graphic/Installation/complete_install.png}	
	\caption{Install finished}
	\label{install_finished}
	\end{figure}
	\FloatBarrier

\end{enumerate}
\newpage
\section{DEVICE}

Device installation is required when  \deviceName have hardware RS 232 / RS 485 / USB to communicate with computer.  Then to correctly communicate  \deviceName with software you should check:
\begin{itemize}
	\item if the USB drivers has been installed correctly (only in  case of USB hardware)
	\item what is number of COM port
\end{itemize}

\subsection{USB DRIVER INSTALATION} \label{usb_driver_installation}

To install drivers for USB hardware follow the steps below:

\begin{enumerate}

	\item Conect \deviceName to USB of computer

	\item Open \textit{Device Manager} - in order to open window \textit{Device Manager} click right button of mouse in  \textit{Windows Start} and select \textit{Device Manager} with list which will be showed.

	\begin{figure}[!h] 
	\centering \includegraphics[width=0.45\textwidth]{Graphic/Installation/open_device_manager.png}	
	\caption{Open device manager}
	\label{open_device_manager}
	\end{figure}
	\FloatBarrier

\item Check If drivers has been installed automatically - when driver has been installed automatically by system, in window\textit{Device Manager} section \textit{Ports (COM \& LPT)} you could found registration: \textit{USB Serial Port (COM x)}. That means, USB driver has been installed successfully.


	\begin{figure}[!h] 
	\centering \includegraphics[width=0.57\textwidth]{Graphic/Installation/driver_usb_installed.png}	
	\caption{Driver USB installed}
	\label{driver_USB_installed}
	\end{figure}
	\FloatBarrier

	Otherwise you see exclamation mark beside not installed device. Then you have to install USB driver manually.

	\begin{figure}[!h] 
	\centering \includegraphics[width=0.57\textwidth]{Graphic/Installation/driver_usb_no_installed.png}	
	\caption{Driver USB no installed}
	\label{driver_USB_ no_installed}
	\end{figure}
	\FloatBarrier

\item Install driver USB manually - in order to install driver USB manually follow the steps below. 
	\begin{enumerate}
		\item select USB device to install - select device to install driver by click right mouse button. From popup menu choose \textit{Update Driver Software}.

		\begin{figure}[!h] 
		\centering \includegraphics[width=0.7\textwidth]{Graphic/Installation/sellect_USB_device_to_install.png}	
		\caption{Select USB device to install}
		\label{select_USB_device_ to_install}
		\end{figure}
		\FloatBarrier

\item choose option \textit{Browse my computer for update driver software}

		\begin{figure}[!h] 
		\centering \includegraphics[width=0.7\textwidth]{Graphic/Installation/install_driver_USB.png}	
		\caption{Choose manually option}
		\label{choose_manually_option}
		\end{figure}
		\FloatBarrier

\item select folder that contains files driver for \deviceName. Driver files has been installed during software installation. You can find him in program directory (default: \\ \textit{c:/Program Files (x86)/Quartz Monitor/Drivers})

	\begin{figure}[!h] 
		\centering \includegraphics[width=0.7\textwidth]{Graphic/Installation/select_driver_path.png}	
		\caption{Select location files driver of USB}
		\label{select_driver_USB_location}
		\end{figure}
		\FloatBarrier

\item confirm location of USB driver files by click \textit{Next}

	\begin{figure}[!h] 
		\centering \includegraphics[width=0.7\textwidth]{Graphic/Installation/confirm_driver_path.png}	
		\caption{Confirm driver USB location}
		\label{confirm_driver_USB_location}
		\end{figure}
		\FloatBarrier


\item finished

	\begin{figure}[!h] 
		\centering \includegraphics[width=0.45\textwidth]{Graphic/Installation/install_driver_finished.png}	
		\caption{Install driver USB finished}
		\label{install_driver_USB_finished}
		\end{figure}
		\FloatBarrier

	\end{enumerate}

\end{enumerate}

\subsection {CHECK COM PORT NUMBER} \label{check_port_com}

	In order to read port COM number which was set automatically by system for \deviceName during connected interface of device to computer you have to open window \textit{Device Manager} (how open \textit{Device Manager} see chapter \textit{3.2.1 USB Driver Installation}) and read from section \textit{Port (COM \& LPT)} number of COM. Number of COM is required to correctly fill communication parameters during configuration device for serial port interface.

	\begin{figure}[!h] 
	\centering \includegraphics[width=0.67\textwidth]{Graphic/Installation/check_port_number.png}	
	\caption{Check number port COM}
	\label{check_number_port_COM}
	\end{figure}
	\FloatBarrier

COM port number you find in bracket beside text \textit{USB Serial Port}. From example above you can recognize port as \textbf{COM4}.